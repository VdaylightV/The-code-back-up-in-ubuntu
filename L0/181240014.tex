\documentclass[UTF8]{ctexart}
\title{L0实验报告}
\author{甘晨 181240014}
\date{\today}
\begin{document}
\maketitle
\subsection{游戏说明}
这个游戏的设计思路非常简单,基本思路就是去满足实验指南对游戏的要求,例如有肉眼可见的图形界面,能实现和用户的交互,以及按下Esc键后游戏终止。

这个游戏实际上是一个屏幕填充游戏,make run之后qemu上会出现一个带有白色边框的画面,画面中间是黑色的,玩家按动键盘的任意键(除去Esc键),黑色部分会从右上角开始向前填充,玩家通过不断按键可以填充整个屏幕。当整个屏幕都被填充为黄色之后,屏幕又会被重置为黑色,玩家可以继续填充,如此循环往复下去,其中当玩家不想玩时,随时可以通过按Esc键退出。
\subsection{实验感想}
实验L0中用到的大部分知识在去年的PA中似乎都有用到,例如从键盘获取输入以及填充屏幕,感觉回顾一下PA的相关实验过程,再通过调用Abstract-machine中提供的API,基本上就可以完成实验任务了。
\end{document}
